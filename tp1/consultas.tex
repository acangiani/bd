\newpage
% cambiar título?
\section{Consultas}
% Link a la documentación de listing: https://en.wikibooks.org/wiki/LaTeX/Source_Code_Listings#Using_the_listings_package
% creo que podríamos tocarle un poco el estilo si hay tiempo pero resuelve el problema.

\begin{enumerate}
\item El listado de inscriptos en cada categoría para el armado de llaves:
\lstinputlisting[language=SQL, firstline=3, lastline=39]{consultas.sql}

\item El país que obtuvo mayor cantidad de medallas de oro, plata y bronce:
\lstinputlisting[language=SQL, firstline=44, lastline=62]{consultas.sql}

\item Sabiendo que las medallas de oro suman 3 puntos, las de plata 2 y las de bronce 1 punto, se quiere realizar un ranking de puntaje por país y otro por escuela:
\lstinputlisting[language=SQL, firstline=66, lastline=116]{consultas.sql}

\item Dado un competidor, la lista de categorías donde haya participado y el resultado obtenido:
\lstinputlisting[language=SQL, firstline=120, lastline=134]{consultas.sql}

\item El medallero por escuela:
\lstinputlisting[language=SQL, firstline=137, lastline=148]{consultas.sql}

\item El listado de los árbitros por país:
\lstinputlisting[language=SQL, firstline=151, lastline=153]{consultas.sql}

\item La lista de todos los árbitros que actuaron como árbitro central en las modalidades de combate:
\lstinputlisting[language=SQL, firstline=156, lastline=166]{consultas.sql}

\item La lista de equipos por país:
\lstinputlisting[language=SQL, firstline=169, lastline=175]{consultas.sql}

\end{enumerate}
